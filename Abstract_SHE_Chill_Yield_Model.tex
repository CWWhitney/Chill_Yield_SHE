\documentclass[]{article}
\usepackage{lmodern}
\usepackage{amssymb,amsmath}
\usepackage{ifxetex,ifluatex}
\usepackage{fixltx2e} % provides \textsubscript
\ifnum 0\ifxetex 1\fi\ifluatex 1\fi=0 % if pdftex
  \usepackage[T1]{fontenc}
  \usepackage[utf8]{inputenc}
\else % if luatex or xelatex
  \ifxetex
    \usepackage{mathspec}
  \else
    \usepackage{fontspec}
  \fi
  \defaultfontfeatures{Ligatures=TeX,Scale=MatchLowercase}
\fi
% use upquote if available, for straight quotes in verbatim environments
\IfFileExists{upquote.sty}{\usepackage{upquote}}{}
% use microtype if available
\IfFileExists{microtype.sty}{%
\usepackage{microtype}
\UseMicrotypeSet[protrusion]{basicmath} % disable protrusion for tt fonts
}{}
\usepackage[margin=1in]{geometry}
\usepackage{hyperref}
\hypersetup{unicode=true,
            pdftitle={Chill Portions and yield using actual data},
            pdfauthor={Katja Schiffersa, Cory Whitneya, Eduardo Fernandeza and Eike Luedelinga   aINRES-Horticultural Sciences, University of Bonn, Auf dem Huegel 6, 53121 Bonn, Germany},
            pdfborder={0 0 0},
            breaklinks=true}
\urlstyle{same}  % don't use monospace font for urls
\usepackage{graphicx,grffile}
\makeatletter
\def\maxwidth{\ifdim\Gin@nat@width>\linewidth\linewidth\else\Gin@nat@width\fi}
\def\maxheight{\ifdim\Gin@nat@height>\textheight\textheight\else\Gin@nat@height\fi}
\makeatother
% Scale images if necessary, so that they will not overflow the page
% margins by default, and it is still possible to overwrite the defaults
% using explicit options in \includegraphics[width, height, ...]{}
\setkeys{Gin}{width=\maxwidth,height=\maxheight,keepaspectratio}
\IfFileExists{parskip.sty}{%
\usepackage{parskip}
}{% else
\setlength{\parindent}{0pt}
\setlength{\parskip}{6pt plus 2pt minus 1pt}
}
\setlength{\emergencystretch}{3em}  % prevent overfull lines
\providecommand{\tightlist}{%
  \setlength{\itemsep}{0pt}\setlength{\parskip}{0pt}}
\setcounter{secnumdepth}{0}
% Redefines (sub)paragraphs to behave more like sections
\ifx\paragraph\undefined\else
\let\oldparagraph\paragraph
\renewcommand{\paragraph}[1]{\oldparagraph{#1}\mbox{}}
\fi
\ifx\subparagraph\undefined\else
\let\oldsubparagraph\subparagraph
\renewcommand{\subparagraph}[1]{\oldsubparagraph{#1}\mbox{}}
\fi

%%% Use protect on footnotes to avoid problems with footnotes in titles
\let\rmarkdownfootnote\footnote%
\def\footnote{\protect\rmarkdownfootnote}

%%% Change title format to be more compact
\usepackage{titling}

% Create subtitle command for use in maketitle
\providecommand{\subtitle}[1]{
  \posttitle{
    \begin{center}\large#1\end{center}
    }
}

\setlength{\droptitle}{-2em}

  \title{Chill Portions and yield using actual data}
    \pretitle{\vspace{\droptitle}\centering\huge}
  \posttitle{\par}
    \author{Katja Schiffers\textsuperscript{a}, Cory Whitney\textsuperscript{a},
Eduardo Fernandez\textsuperscript{a} and Eike
Luedeling\textsuperscript{a} \textsuperscript{a}INRES-Horticultural
Sciences, University of Bonn, Auf dem Huegel 6, 53121 Bonn, Germany}
    \preauthor{\centering\large\emph}
  \postauthor{\par}
    \date{}
    \predate{}\postdate{}
  
\usepackage{booktabs}
\usepackage{longtable}
\usepackage{array}
\usepackage{multirow}
\usepackage{wrapfig}
\usepackage{float}
\usepackage{colortbl}
\usepackage{pdflscape}
\usepackage{tabu}
\usepackage{threeparttable}
\usepackage{threeparttablex}
\usepackage[normalem]{ulem}
\usepackage{makecell}
\usepackage{xcolor}

\begin{document}
\maketitle

Data is often limited for assessing the relationship between
temperatures and yield. Herw we show how, despite this lack of data, we
may still be able to make assessments and produce useful projections for
farmers and decision makers. With the right tools this data limitation
does not need to hinder our abilities to assess the relationships
between temperature and yield. For coarse assessments a lot of data may
not be necessary. We use the pasitR package (Schiffers et al., 2018) in
the R programming language (R Core Team, 2019) to illustrate methods
whereby we can embrace the inherent uncertainty in such assessments to
overcome the need for preciseness. We show a potential method for
dealing with important but also necessarily uncertain relationships in
model forecasts.

\hypertarget{yield-and-chill-data}{%
\subsection{Yield and chill data}\label{yield-and-chill-data}}

We offer an example of assessing yield given chill for sweet cherries
(\emph{Prunus avium} L.) `Lapins' and `Brooks' varieties. We applied
procedures from the \texttt{pasitR} library for estimating yield as a
function of chill accumulation (Schiffers et al., 2018). The data was
provided by the experimental orchard of the School of Agronomy at the
Pontificia Universidad Catolica de Valparaiso (Table 1). We used weather
data obtained from a local weather station (Table 2).

We used the \texttt{tempResponse\_daily\_list} in the \texttt{chillR}
package (Luedeling, 2019) to compute the chill accumulation for each
season. We defined the chilling season as the period between
1\textsuperscript{st} of May and 31\textsuperscript{st} of August (Table
3).

\begin{table}

\caption{\label{tab:tables_1_3}Yield records (in tons per hectare) for 8 seasons (2010 to 2017) for two sweet cherry cultivars (Lapins and Brooks).}
\centering
\begin{tabular}[t]{r|l|r}
\hline
Year & Variety & Yield\\
\hline
2010 & Lapins & 16.387600\\
\hline
2011 & Lapins & 11.401600\\
\hline
2012 & Lapins & 1.599200\\
\hline
2013 & Lapins & 13.521200\\
\hline
2014 & Lapins & 21.648480\\
\hline
2015 & Lapins & 9.413200\\
\hline
2016 & Lapins & 24.974440\\
\hline
2017 & Lapins & 8.515682\\
\hline
2010 & Brooks & 6.412400\\
\hline
2011 & Brooks & 1.296000\\
\hline
2012 & Brooks & 1.032000\\
\hline
2013 & Brooks & 3.396800\\
\hline
2014 & Brooks & 6.872400\\
\hline
2015 & Brooks & 2.887160\\
\hline
2016 & Brooks & 9.217320\\
\hline
2017 & Brooks & 9.892581\\
\hline
\end{tabular}
\end{table}

\begin{table}

\caption{\label{tab:tables_1_3}Weather data from a weather station placed in the orchard.}
\centering
\begin{tabular}[t]{r|l|r|r|r|r|r|r}
\hline
YEARMODA & Weather\_Station & Year & Month & Day & JDay & Tmin & Tmax\\
\hline
20100101 & Quillota & 2010 & 1 & 1 & 1 & 6.2 & 31.0\\
\hline
20100102 & Quillota & 2010 & 1 & 2 & 2 & 7.6 & 29.4\\
\hline
20100103 & Quillota & 2010 & 1 & 3 & 3 & 12.2 & 23.2\\
\hline
20100104 & Quillota & 2010 & 1 & 4 & 4 & 8.0 & 24.0\\
\hline
20100105 & Quillota & 2010 & 1 & 5 & 5 & 8.0 & 26.5\\
\hline
20100106 & Quillota & 2010 & 1 & 6 & 6 & 9.0 & 27.0\\
\hline
20100107 & Quillota & 2010 & 1 & 7 & 7 & 8.5 & 24.0\\
\hline
20100108 & Quillota & 2010 & 1 & 8 & 8 & 7.0 & 28.0\\
\hline
20100109 & Quillota & 2010 & 1 & 9 & 9 & 8.0 & 28.0\\
\hline
20100110 & Quillota & 2010 & 1 & 10 & 10 & 8.0 & 24.0\\
\hline
\end{tabular}
\end{table}

\begin{table}

\caption{\label{tab:tables_1_3}Computed chill accumulation for each pre-defined chilling season between May 1 and August 31.}
\centering
\begin{tabular}[t]{l|r|r|r|r|r}
\hline
Season & End\_year & Season\_days & Data\_days & Perc\_complete & Chill\_Portions\\
\hline
2009/2010 & 2010 & 123 & 123 & 100 & 68.45256\\
\hline
2010/2011 & 2011 & 123 & 123 & 100 & 60.46397\\
\hline
2011/2012 & 2012 & 123 & 123 & 100 & 44.61716\\
\hline
2012/2013 & 2013 & 123 & 123 & 100 & 52.14200\\
\hline
2013/2014 & 2014 & 123 & 123 & 100 & 54.60832\\
\hline
2014/2015 & 2015 & 123 & 123 & 100 & 43.24068\\
\hline
2015/2016 & 2016 & 123 & 123 & 100 & 53.47477\\
\hline
2016/2017 & 2017 & 123 & 123 & 100 & 53.95999\\
\hline
\end{tabular}
\end{table}

We used the \texttt{chillscatter} function to create a scatter plot of
Chill Portions and yield. We calculated the associated estimated
densities with loess smooth linear fits density curves using the
\texttt{scatter.hist} function in the \texttt{plyr} package (Wickham,
2019).

\begin{figure}
\centering
\includegraphics{Abstract_SHE_Chill_Yield_Model_files/figure-latex/unnamed-chunk-2-1.pdf}
\caption{Scatter plot of Chill Portions (x) and yield (y) for sweet
cherries.}
\end{figure}

We used \texttt{chillkernel} to perform a two-dimensional kernel density
estimation for yield and chill using the \texttt{kde2d} function in the
\texttt{MASS} package (Ripley, 2019). The density function restricts the
shape of the kernel to a bivariate normal kernel, so this looks slightly
different compared to the scatter plot estimates above.

\begin{figure}
\centering
\includegraphics{Abstract_SHE_Chill_Yield_Model_files/figure-latex/unnamed-chunk-3-1.pdf}
\caption{Kernel density matrix of Chill Portions (x) and yield (y) for
sweet cherries.}
\end{figure}

\texttt{chillkernel} shows a density surface plot of Chill Portions (x)
and yield (y). The legend shows the value for the estimated density (z).
The plot is made with the \texttt{filled.contour} function of the
\texttt{graphics} package (R Core Team, 2019).

In \texttt{chillkernel} the density (z) over the entire plot integrates
to one, and therefore represents the relative probability of an
observation (yield along y-axis) given a specific chill portion (along
x-axis).

\hypertarget{estimated-yield-given-the-expected-chill}{%
\subsection{Estimated yield given the expected
chill}\label{estimated-yield-given-the-expected-chill}}

We used the \texttt{pasitR} function \texttt{chillkernelslice} to
calculate the estimated yield given the expected chill, based on a slice
of `z' from the Kernel density calculated with \texttt{chillkernel}. The
\texttt{expectedchill} parameter is set to 30.

\begin{figure}
\centering
\includegraphics{Abstract_SHE_Chill_Yield_Model_files/figure-latex/unnamed-chunk-4-1.pdf}
\caption{Estimated yield of sweet cherry given the expected chill, based
on a slice of `z' from the Kernel density.}
\end{figure}

The \texttt{chillkernelslice} function plots the probabilities (shown
along the y-axis) for the expected yield (shown along the x-axis). Since
this is a cut through the density kernel \texttt{chillkernel}, which
integrates to 1, the probability values are relative, not absolute
measures.

\hypertarget{chill-portion-intervals}{%
\subsection{Chill portion intervals}\label{chill-portion-intervals}}

We used the \texttt{chillviolin} function to determine possible Chill
Portion intervals by calculating the optimal interval width for Chill
Portions using the \texttt{IQR} function in the \texttt{stats} package,
after the Freedman-Diaconis rule (IQR = interquartile range) (R Core
Team, 2019). Plot made with \texttt{ggplot2} (Wickham et al., 2019).

\begin{figure}
\centering
\includegraphics{Abstract_SHE_Chill_Yield_Model_files/figure-latex/unnamed-chunk-5-1.pdf}
\caption{Violin plots with boxplot overlays of possible Chill Portions
(x) and yield (y) with six different intervals of Chill Portions.}
\end{figure}

\hypertarget{probability-of-yield-given-chill}{%
\subsection{Probability of yield given
chill}\label{probability-of-yield-given-chill}}

We use the \texttt{chillkernelslicerange} to show the probable yield
given a likely range of expected Chill Portions. The optimized
interquartile ranges for Chill Portion intervals (shown in
\texttt{chillviolin}) can be used to select a range to slice from the
density kernel \texttt{chillkernel} as was done for a single chill value
in \texttt{chillkernelslice}. As with \texttt{chillkernelslice} the
probability values shown are relative, not absolute measures. They are
the result of cuts through the density kernel \texttt{chillkernel},
which integrates to 1.

\begin{figure}
\centering
\includegraphics{Abstract_SHE_Chill_Yield_Model_files/figure-latex/unnamed-chunk-6-1.pdf}
\caption{Probabilities (shown along the y-axis) for the expected yield
(shown along the x-axis). Here we set the minimum Chill Portions to 53
and the maximum to 57.}
\end{figure}

\hypertarget{next-steps}{%
\section{Next steps}\label{next-steps}}

The \texttt{pasitR} functions closely follow chillR (Luedeling, 2019)
and decisionSupport (Luedeling et al., 2019). We will continue to
develop these and may intergrate them into future version of these
packages. The functions are all stored in an open access repository
(\url{https://github.com/hortibonn/pasitR}) and are free to use and
modify by all (Schiffers et al., 2018).

\hypertarget{references}{%
\subsection*{References}\label{references}}
\addcontentsline{toc}{subsection}{References}

\hypertarget{refs}{}
\leavevmode\hypertarget{ref-R-chillR}{}%
Luedeling, E. 2019. ChillR: Statistical methods for phenology analysis
in temperate fruit trees.

\leavevmode\hypertarget{ref-R-decisionSupport}{}%
Luedeling, E., Goehring, L. and Schiffers, K. 2019. DecisionSupport:
Quantitative support of decision making under uncertainty.

\leavevmode\hypertarget{ref-R-base}{}%
R Core Team. 2019. R: A language and environment for statistical
computing. Vienna, Austria: R Foundation for Statistical Computing.

\leavevmode\hypertarget{ref-R-MASS}{}%
Ripley, B. 2019. MASS: Support functions and datasets for venables and
ripley's mass.

\leavevmode\hypertarget{ref-R-pasitR}{}%
Schiffers, K., Whitney, C., Fernandez, E. and Luedeling, E. 2018.
PasitR: Calculates common functions for the pasit project.

\leavevmode\hypertarget{ref-R-plyr}{}%
Wickham, H. 2019. Plyr: Tools for splitting, applying and combining
data.

\leavevmode\hypertarget{ref-R-ggplot2}{}%
Wickham, H., Chang, W., Henry, L., Pedersen, T.L., Takahashi, K., Wilke,
C., Woo, K. and Yutani, H. 2019. Ggplot2: Create elegant data
visualisations using the grammar of graphics.


\end{document}
